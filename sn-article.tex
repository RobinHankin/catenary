%%\documentclass[referee,sn-basic]{sn-jnl}% referee option is meant for double line spacing
%%\documentclass[lineno,pdflatex,sn-basic]{sn-jnl}% Basic Springer Nature Reference Style/Chemistry Reference Style

\documentclass[pdflatex,sn-mathphys-num]{sn-jnl}% Math and Physical Sciences Numbered Reference Style

\usepackage{graphicx}%
\usepackage{multirow}%
\usepackage{amsmath,amssymb,amsfonts}%
\usepackage{amsthm}%
\usepackage{mathrsfs}%
\usepackage[title]{appendix}%
\usepackage{xcolor}%
\usepackage{textcomp}%
\usepackage{manyfoot}%
\usepackage{booktabs}%
\usepackage{algorithm}%
\usepackage{algorithmicx}%
\usepackage{algpseudocode}%
\usepackage{listings}%


\theoremstyle{thmstyleone}%
\newtheorem{theorem}{Theorem}%  meant for continuous numbers
%%\newtheorem{theorem}{Theorem}[section]% meant for sectionwise numbers
%% optional argument [theorem] produces theorem numbering sequence instead of independent numbers for Proposition
\newtheorem{proposition}[theorem]{Proposition}% 
%\newtheorem{proposition}{Proposition}% to get separate numbers for theorem and proposition etc.

\theoremstyle{thmstyletwo}%
\newtheorem{example}{Example}%
\newtheorem{remark}{Remark}%

\theoremstyle{thmstylethree}%
\newtheorem{definition}{Definition}%

\raggedbottom
%%\unnumbered% uncomment this for unnumbered level heads

\begin{document}
\newcommand{\di}[2]{\frac{\partial{#1}}{\partial{#2}}}
\newcommand{\dii}[3]{\frac{\partial^2{#1}}{\partial{#2}\partial{#3}}}
\newcommand{\diii}[4]{\frac{\partial^3{#1}}{\partial{#2}\partial{#3}\partial{#4}}}
\newcommand{\el}[1]{\frac{d}{dx}\di{#1}{y'} - \di{#1}{y}}
\title[Article Title]{Article Title}

%%=============================================================%%
%% GivenName	-> \fnm{Joergen W.}
%% Particle	-> \spfx{van der} -> surname prefix
%% FamilyName	-> \sur{Ploeg}
%% Suffix	-> \sfx{IV}
%% \author*[1,2]{\fnm{Joergen W.} \spfx{van der} \sur{Ploeg} 
%%  \sfx{IV}}\email{iauthor@gmail.com}
%%=============================================================%%

\author{\fnm{Robin} \sur{Hankin}}\email{hankin.robin@gmail.com}

\affil{\orgdiv{Computer and Mathematical Sciences}, \orgname{University of Stirling}, \city{Stirling}, \postcode{FK9 4LA}, \country{Scotland}}




\abstract{ This short document presents a method for dealing with
  slight perturbations to functionals.  We apply the Euler-Lagrange
  equations to a slightly perturbed functional and retain only terms to first order.

  Three applications from different areas of mathematical physics are
  presented.

  }

\keywords{keyword1, Keyword2, Keyword3, Keyword4}

%%\pacs[JEL Classification]{D8, H51}

%%\pacs[MSC Classification]{35A01, 65L10, 65L12, 65L20, 65L70}

\maketitle

\section{Introduction}\label{sec1}


The Euler-Lagrange equations were developed by Leonhard Euler and
Joseph-Louis Lagrange in the 1750s.  Following \citep{jeffreys1972},
suppose we have an integral of the form

\begin{equation}
F[y] = \int_{x_1}^{x_2} F(x,y(x),y'(x))\,dx
\end{equation}

\noindent where $F$ is a given functional and $y$ a function of $x$,
to be determined.  We wish to minimize, over a space of suitably
well-behaved functions
$y\colon\left[x_1,x_2\right]\longrightarrow\mathbb{R}$, the value of
$F[y]$.  Now suppose we have identified such a $y$ that locally
minimizes $F[y]$, and we now consider a perturbed functional
$\mathcal{F}=F+\epsilon G$.  We take $\epsilon$ to be sufficiently
small for us to neglect second and higher powers.  Can we similarly
perturb $y$ (to $y+\epsilon z$, say), so as to minimize $F+\epsilon
G$?

Under these circumstances, the perturbation term $z=z(x)$ is
interesting in its own right, and the fact the perturbation is linear
leads to new insight.

Formally, we suppose:

\begin{eqnarray}
  \frac{d}{dx}\di{F}{y'}-\di{F}{y} &=& 0\nonumber\\
  \frac{d}{dx}\di{(F+\epsilon G)}{y'}-\di{(F+\epsilon G)}{y} &=& 0
\end{eqnarray}

We have, working to first order:

\begin{eqnarray}
\frac{d}{dx}\left[\di{F}{y'}\right]
&=& \di{F}{y}\nonumber\\
\frac{d}{dx}\left[\left.\di{(F+\epsilon G)}{y'}\right|_{x,y+\epsilon z,y'+\epsilon z'}\right]
&=& \left.\di{(F+\epsilon G)}{y}\right|_{x,y+\epsilon z,y'+\epsilon z'}\nonumber\\
\frac{d}{dx}\left[\di{}{y'}\left(F + \epsilon z\di{F}{y} + \epsilon z'\di{F}{y'}+\epsilon G\right)\right]
&=& \di{}{y}\left(F + \epsilon z\di{F}{y} + \epsilon z'\di{F}{y'}+\epsilon G\right)\nonumber\\
\frac{d}{dx}\left[\di{F}{y'} + \epsilon\left(z\dii{F}{y}{y'} + z\dii{F}{y'}{y'} + \di{G}{y'}\right)\right]
&=& \di{F}{y} + \epsilon\left(z\dii{F}{y}{y} + z'\dii{F}{y}{y'} + \di{G}{y}\right)\nonumber\\
\frac{d}{dx}\left[z\dii{F}{y}{y'} + z'\dii{F}{y'}{y'} + \di{G}{y'}\right]
&=& z\dii{F}{y}{y} + z'\dii{F}{y}{y'} + \di{G}{y}
\end{eqnarray}

Now we identify $\frac{d}{dx}$ with $\di{}{x} + y'\di{}{y} +
y''\di{}{y'} + z'\di{}{z} + z''\di{}{z'}$.  In the interests of
typographical convenience we revert to inline notation, viz
$\di{F}{y}\equiv\partial_yF\equiv F_y$.


\begin{eqnarray}
\frac{d}{dx}\left[zF_{yy'} + z'F_{y'y'} + G_{y'}\right]
&=& zF_{yy} + z'F_{yy'} + G_{y}\nonumber\\
\left(\partial_{x} + y'\partial_{y} + y''\partial_{y'} + z'\partial_{z} + z''\partial_{z'}\right)
\left[zF_{yy'} + z'F_{y'y'} + G_{y'}\right]
&=& zF_{yy} + z'F_{yy'} + G_{y}\nonumber\\
(zF_{xyy'} + z'F_{xy'y'} + G_{xy'})
+ (zy'F_{yyy'} + z'y'F_{yy'y'} + y'G_{yy'})\nonumber\\
+ ( zy''F_{yy'y'} + z'y''F_{y'y'y'} + y''G_{y'y'})
+ (z'F_{yy'} + z''F_{y'y'})
&=& zF_{yy} + z'F_{yy'} + G_{y}\nonumber\\
\end{eqnarray}

Collecting terms:

\begin{eqnarray}
&{}& z''F_{y'y'}\\
&+& z'(F_{xy'y'} + y'F_{yy'y'} + y''F_{y'y'y'})\\
&+& z (F_{xyy'} + y'F_{yyy'} + y''F_{yy'}-F_{yy})\\
&+& (G_{xy'} + y'G_{yy'} + y''G_{y'y'}- G_{y})\\
= 0
\end{eqnarray}

(above we cancel the $z'F_{yy'}$ term). This is a second order ODE in
$z$, which gives the first-order perturbation $z$ so that $y$
maximizes $F$ and $y+\epsilon z$ maximizes $F+\epsilon G$.

\bibliography{sn-bibliography}% common bib file
%% if required, the content of .bbl file can be included here once bbl is generated
%%\input sn-article.bbl

\end{document}
