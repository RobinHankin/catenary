%%\documentclass[referee,sn-basic]{sn-jnl}% referee option is meant for double line spacing
%%\documentclass[lineno,pdflatex,sn-basic]{sn-jnl}% Basic Springer Nature Reference Style/Chemistry Reference Style

\documentclass[pdflatex,sn-mathphys-num]{sn-jnl}% Math and Physical Sciences Numbered Reference Style

\usepackage{graphicx}%
\usepackage{multirow}%
\usepackage{amsmath,amssymb,amsfonts}%
\usepackage{amsthm}%
\usepackage{mathrsfs}%
\usepackage[title]{appendix}%
\usepackage{xcolor}%
\usepackage{textcomp}%
\usepackage{manyfoot}%
\usepackage{booktabs}%
\usepackage{algorithm}%
\usepackage{algorithmicx}%
\usepackage{algpseudocode}%
\usepackage{listings}%


\usepackage{siunitx}


\theoremstyle{thmstyleone}%
\newtheorem{theorem}{Theorem}%  meant for continuous numbers
%%\newtheorem{theorem}{Theorem}[section]% meant for sectionwise numbers
%% optional argument [theorem] produces theorem numbering sequence instead of independent numbers for Proposition
\newtheorem{proposition}[theorem]{Proposition}% 
%\newtheorem{proposition}{Proposition}% to get separate numbers for theorem and proposition etc.

\theoremstyle{thmstyletwo}%
\newtheorem{example}{Example}%
\newtheorem{remark}{Remark}%

\theoremstyle{thmstylethree}%
\newtheorem{definition}{Definition}%

\raggedbottom
%%\unnumbered% uncomment this for unnumbered level heads

\begin{document}
\newcommand{\di}[2]{\frac{\partial{#1}}{\partial{#2}}}
\newcommand{\dii}[3]{\frac{\partial^2{#1}}{\partial{#2}\partial{#3}}}
\newcommand{\diii}[4]{\frac{\partial^3{#1}}{\partial{#2}\partial{#3}\partial{#4}}}
\newcommand{\el}[1]{\frac{d}{dx}\di{#1}{y'} - \di{#1}{y}}
\title[Linear perturbations of the Euler-Lagrange equation]{Linear perturbations of the Euler-Lagrange equation}

%%=============================================================%%
%% GivenName	-> \fnm{Joergen W.}
%% Particle	-> \spfx{van der} -> surname prefix
%% FamilyName	-> \sur{Ploeg}
%% Suffix	-> \sfx{IV}
%% \author*[1,2]{\fnm{Joergen W.} \spfx{van der} \sur{Ploeg} 
%%  \sfx{IV}}\email{iauthor@gmail.com}
%%=============================================================%%

\author{\fnm{Robin K. S.} \sur{Hankin}}\email{hankin.robin@gmail.com}

\affil{\orgdiv{Computer and Mathematical Sciences}, \orgname{University of Stirling}, \city{Stirling}, \postcode{FK9 4LA}, \country{Scotland}}


\abstract{This short document presents a method for dealing with
  slight perturbations to functionals.  We apply the Euler-Lagrange
  equations to a slightly perturbed functional and retain only terms
  to first order.

  Three applications from different areas of mathematical physics are
  presented.  }

\keywords{Calculus of variations, geometrical optics, catenary,
  inextensible string, Schwarzschild metric}

%%\pacs[JEL Classification]{D8, H51}

%%\pacs[MSC Classification]{35A01, 65L10, 65L12, 65L20, 65L70}

\maketitle

\section{Introduction}

The Euler-Lagrange equations were developed by Leonhard Euler and
Joseph-Louis Lagrange in the 1750s.  Following \citep{jeffreys1972},
suppose we have an integral of the form

\begin{equation}
F[y] = \int_{x_1}^{x_2} F(x,y,y')\,dx
\end{equation}

\noindent where $F$ is a given functional and $y$ a function of $x$,
to be determined.  We wish to minimize the value of $F[y]$, over a
space of suitably well-behaved functions
$y\colon\left[x_1,x_2\right]\longrightarrow\mathbb{R}$.  Now suppose
we have identified such a $y$ that locally minimizes $F[y]$, and we
now consider a perturbed functional $\mathcal{F}=F+\epsilon G$.  We
take $\epsilon$ to be sufficiently small for us to neglect second and
higher powers.  Can we similarly perturb $y$ (to $y+\epsilon z$, say),
so as to minimize $F+\epsilon G$?

Under these circumstances, the perturbation term $z=z(x)$ is
interesting in its own right, and the fact the perturbation is linear
leads to new insight.  Formally, we suppose:

\begin{equation}\label{euler-lagrange}
  \frac{d}{dx}\di{F}{y}-\di{F}{y} = 0,
\end{equation}
%
and further suppose that we have found a function $y=y(x)$ satisfying
equation~\ref{euler-lagrange}.  We now wish to identify a function
$z=z(x)$ so that $y+\epsilon z$ satisfies

\begin{equation}
  \frac{d}{dx}\di{(F+\epsilon G)}{y'}-\di{(F+\epsilon G)}{y} = 0
\end{equation}
%
where $F$ and $G$ are evaluated at $y+\epsilon z$.  We have, working
to first order:

\begin{eqnarray}
\frac{d}{dx}\left[\left.\di{(F+\epsilon G)}{y'}\right|_{x,y+\epsilon z,y'+\epsilon z'}\right]
&=& \left.\di{(F+\epsilon G)}{y}\right|_{x,y+\epsilon z,y'+\epsilon z'}\nonumber\\
\frac{d}{dx}\left[\di{}{y'}\left(F + \epsilon z\di{F}{y} + \epsilon z'\di{F}{y'}+\epsilon G\right)\right]
&=& \di{}{y}\left(F + \epsilon z\di{F}{y} + \epsilon z'\di{F}{y'}+\epsilon G\right)\nonumber\\
\frac{d}{dx}\left[\di{F}{y'} + \epsilon\left(z\dii{F}{y}{y'} + z\dii{F}{y'}{y'} + \di{G}{y'}\right)\right]
&=& \di{F}{y} + \epsilon\left(z\dii{F}{y}{y} + z'\dii{F}{y}{y'} + \di{G}{y}\right)\nonumber\\
\frac{d}{dx}\left[z\dii{F}{y}{y'} + z'\dii{F}{y'}{y'} + \di{G}{y'}\right]
&=& z\dii{F}{y}{y} + z'\dii{F}{y}{y'} + \di{G}{y}.
\end{eqnarray}
%
We now identify $\frac{d}{dx}$ with $\di{}{x} + y'\di{}{y} +
y''\di{}{y'} + z'\di{}{z} + z''\di{}{z'}$.  In the interests of
typographical convenience we revert to inline notation, viz
$\di{F}{y}\equiv\partial_yF\equiv F_y$.

\begin{eqnarray}
\frac{d}{dx}\left[zF_{yy'} + z'F_{y'y'} + G_{y'}\right]
&=& zF_{yy} + z'F_{yy'} + G_{y}\nonumber\\
\left(\partial_{x} + y'\partial_{y} + y''\partial_{y'} + z'\partial_{z} + z''\partial_{z'}\right)
\left[zF_{yy'} + z'F_{y'y'} + G_{y'}\right]
&=& zF_{yy} + z'F_{yy'} + G_{y}\nonumber\\
(zF_{xyy'} + z'F_{xy'y'} + G_{xy'})
+ (zy'F_{yyy'} + z'y'F_{yy'y'} + y'G_{yy'})\nonumber\\
+ ( zy''F_{yy'y'} + z'y''F_{y'y'y'} + y''G_{y'y'})
+ (z'F_{yy'} + z''F_{y'y'})
&=& zF_{yy} + z'F_{yy'} + G_{y}\nonumber\\
\end{eqnarray}

Collecting terms:

\begin{eqnarray}
&{}& z''F_{y'y'}\nonumber\\
&+& z'(F_{xy'y'} + y'F_{yy'y'} + y''F_{y'y'y'})\nonumber\\
&+& z (F_{xyy'} + y'F_{yyy'} + y''F_{yy'}-F_{yy})\nonumber\\
&+& (G_{xy'} + y'G_{yy'} + y''G_{y'y'}- G_{y})\nonumber\\
= 0
\end{eqnarray}
%
(above we cancel the $z'F_{yy'}$ term). This is a second order ODE in
$z$, which gives the first-order perturbation $z$ so that $y+\epsilon
z$ maximizes $F+\epsilon G$.  Alternatively, we may factorize the
expression, giving

\begin{equation}\label{factorized_eqn}
\left(\el{G}\right) + \left(z\di{}{y} + z'\di{}{y'} + z''\di{}{y''}\right)
\left(\el{F}\right)=0.
\end{equation}
%
Similar techniques show that the Beltrami identity,
$F-y'\di{F}{y'}=C$, has a perturbed version
%
\begin{equation}\label{beltrami}
  \left(G-y'\di{G}{y'}\right)
  =
    z \left[y'\frac{\partial^2F}{\partial y\partial y'} + \di{F}{y'}\right]
  + z'\left[y'\frac{\partial^2F}{\partial y'\partial y'}\right] + C.
\end{equation}
%
Equation~\ref{factorized_eqn} is now applied to three simple classical
problems in which a functional is given a small perturbation of some
sort.  The examples are drawn from optics, statics, and the general
theory of relativity.

\section{Examples}

I now present some examples drawn from physics in which the functional
minimized by the Euler-Lagrange equation is given a small, physically
meaningful, perturbation.  I present three cases: ray propagation in
geometric optics, a flexible chain in a gravitational field, and an
idealized light inextensible string in the vicinity of a black hole.
In each case the perturbation of the solution leads to ready insight
into the problem considered.

\subsection{Geometrical optics in a medium of slightly varying refractive index}

We consider two-dimensional ray paths in a medium of refractive index
$1+\phi(x,y)$.  First, we solve the Euler-Lagrange equations exactly.
The paths are given by solving the Euler-Lagrange equations with
functional $F=(1+\phi)\sqrt{1+y'^2}$:

\begin{eqnarray}
  \frac{d}{dx}\di{F}{y'}-\di{F}{y} &=& 0\nonumber\\
  \frac{(\phi_x+y'\phi_y)y'}{\sqrt{1+y'^2}}+\frac{(1+\phi)y''}{(1+y'^2)^{3/2}} -
  \phi_y\sqrt{1+y'^2} &=& 0
  \end{eqnarray}

Simplifying,

\begin{eqnarray}
 (1+\phi) y'' = (1+y'^2)(\phi_y - y'\phi_x).
\end{eqnarray}
%
Except for trivial cases such as $\phi_y=y'\phi_x$, this equation has
no simple (and certainly no intuitively insightful) analytic solution.
We now move to the linearised case, in which we minimize $F+\epsilon G$ with

\begin{equation}F=\sqrt{1+y'^2}\qquad
  G=\phi(x,y)\sqrt{1+y'^2}
\end{equation}

We obtain
\begin{eqnarray}
  \el{F} &=& \frac{y''}{(1+y'^2)^{3/2}}\nonumber\\
  \el{G} &=& \frac{y''\phi}{(1+y'^2)^{3/2}} + \frac{y'\phi_x - \phi_y}{\sqrt{1+y'^2}}
\end{eqnarray}

Thus

\begin{eqnarray}
&{}& \left(z\di{}{y} + z'\di{}{y'} + z''\di{}{y''}\right)\left(\el{F}\right)\nonumber\\
&=&  z'\frac{1}{(1+y'^2)^{3/2}} -z''\frac{3y'}{(1+y'^2)^{5/2}}
\end{eqnarray}

whence equation~\ref{factorized_eqn} becomes

\begin{equation}
    \frac{z''}{(1+y'^2)^{3/2}}
  - \frac{3y'y''}{(1+y'^2)^{3/2}}
  + \frac{\phi_xy'}{\sqrt{1+y'^2}}
  + \frac{\phi_yy'^2}{\sqrt{1+y'^2}}
  + \frac{y''}{(1+y'^2)^{3/2}}-\phi_y\sqrt{1+y'^2}
\end{equation}

We recall that $y'' = 0$ in the unperturbed case, implying that $y'$
is constant.  Writing $y'= s$ we obtain

\begin{equation}\nonumber
  \frac{z''}{(1+s^2)^{3/2}} +
  \frac{\phi_xs}{\sqrt{1+s^2}} + \frac{\phi_ys^2}{\sqrt{1+s^2}} -\phi_y\sqrt{1+s^2}
\end{equation}

\begin{equation}\label{zdashdashmirage}
  z'' = (1+s^2)(\phi_y-\phi_xs)
\end{equation}

We may use this as a model for refraction of light rays in a medium in
which the refractive index typically differs from unity only slightly;
examples might include atmospheric mirages~\cite{trankle1999}, and the
refraction of radio waves in interstellar space~\cite{romani1986}.  We
may infer directly from equation~\ref{zdashdashmirage} that the rays
(and also ray slopes, interesting in the context of imaging) deviate
from unperturbed straight-line rays {\em linearly} with excess
refractive index.


%\begin{figure}[h]
%\centering
%\includegraphics[width=0.9\textwidth]{euler_lagrange_perturbed_files/figure-latex/plotdelta-1.pdf}
%\caption{Light rays following geometrical optics}\label{plotdelta}
%\end{figure}
%
%\begin{figure}[h]
%\centering
%\includegraphics[width=0.9\textwidth]{euler_lagrange_perturbed_files/figure-latex/plotdeltacloseup-1.pdf}
%\caption{Light rays following geometrical optics}\label{plotdeltacloseup}
%\end{figure}


\subsection{Catenary}

A flexible uniform chain supported at two points in a uniform
gravitational field takes the form of a catenary, a hyperbolic cosine.
This may be established by applying to Euler-Lagrange equation to
functional $F=\sqrt{1+y'^2}(y+\Lambda)$ where $\Lambda$ is a Lagrange
multiplier that controls the length of the chain.

\begin{eqnarray}
  \frac{d}{dx}\di{F}{y'} - \di{F}{y}
  &=& 
  \frac{d}{dx}\left[\frac{(y+\Lambda)y'}{\sqrt{1+y'^2}}\right] -  \frac{1}{\sqrt{1+y'^2}}\nonumber\\
&=&   \frac{y'^2}{\sqrt{1+y'^2}} + \frac{(y+\Lambda)y''}{(1+y'^2)^{3/2}} - \sqrt{1+y'^2}\nonumber\\
  &=&   \frac{(y+\Lambda)y''}{(1+y'^2)^{3/2}} - \frac{1}{\sqrt{1+y'^2}}
  \label{catenary_EL_f}
\end{eqnarray}

Rearranging equation~\ref{catenary_EL_f} gives

\begin{eqnarray}
  (y+\Lambda)y'' &=& 1+y'^2\label{catenary_ODE}
\end{eqnarray}

Equation~\ref{catenary_ODE} requires three constants for solution
(viz, two constants of integration arising from the fact that it is a
second order ODE, and the Lagrange multiplier $\Lambda$).  In
practice, these constants are determined by constraining three
physical features of the chain, typically the height of the curve at
two points, $y(x_1)$, $y(x_2)$ and the length of the chain between
them, $\int_{x_1}^{x_2}\sqrt{1+y'^2}\,dx$.  For ease of exposition we
require the chain to pass through $(\pm 2,\cosh 2)$ and specify an arc
length of $e+e^{-1}$, giving $\Lambda=0$ whence $y=\cosh x$.

We are now in a position to assess the perturbation from the standard
catenary caused by small perturbations to the gravitational field
potential.  We will consider a gravitational potential of $y +
\phi(x,y)$, giving functionals

\begin{eqnarray}\label{FandG}
F = \sqrt{1+y'^2}(y + \Lambda)\qquad G =  \sqrt{1+y'^2}(\phi + \lambda)
\end{eqnarray}
%
where $\phi=\phi(x,y)$ is a small perturbation.  Here 

\begin{eqnarray}
&{}& z''F_{y'y'}\nonumber\\
&+& z'(F_{xy'y'} + y'F_{yy'y'} + y''F_{y'y'y'})\nonumber\\
&+& z (F_{xyy'} + y'F_{yyy'} + y''F_{yy'}-F_{yy})\nonumber\\
&+& (G_{xy'} + y'G_{yy'} + y''G_{y'y'}- G_{y})\nonumber\\
= 0\\
&{}&  z''\left(\frac{y+\Lambda}{(1+y'^2)^{3/2}}\right)\nonumber\\
&+& z' \left(-\frac{y'}{(1+y'^2)^{3/2}} -\frac{3(y+\Lambda)y'y''}{(1+y'^2)^{5/2}}\right)\nonumber\\
&+& z  \left(\frac{y''}{(1+y'^2)^{3/2}}\right)\nonumber\\
&+& \frac{y'\phi_x-\phi_y}{\sqrt{1+y'^2}} + \frac{(\phi+\lambda)y''}{(1+y'^2)^{3/2}}=0
\end{eqnarray}
%
Now we may use the unperturbed solution of $y=\cosh x$, giving
$y=y''=\sqrt{1+y'^2}=\cosh x$ and $y'=\sinh x$.  After simplification:

\begin{equation}\label{pertcan}
  z'' - 4z'\tanh x - z =
  \phi + \lambda + \phi_x\sinh x\cosh x - \phi_y\cosh x
\end{equation}
%
Equation~\ref{pertcan} is readily solved numerically.  Three unknowns
are required: two constants of integration, as \ref{pertcan} is a
second order ODE; and the value of Lagrange's undetermined multiplier
$\lambda$.  In this case we have three constraints corresponding to
the perturbation at the supports being zero, $z(\pm 2)=0$; and in
addition we know that the perturbed solution has the same arc length
as the unperturbed solution, for the chain is inextensible.

As an example we will consider an inextensible chain at rest when
viewed in a coordinate system $(x,y,z)$ with unit vectors
$\boldsymbol{e}_x,\boldsymbol{e}_y,\boldsymbol{e}_z$ rotating with
angular velocity~$\boldsymbol{\omega}$.  Objects at rest in this frame
behave as though they are subject to a body force of
$\boldsymbol{\omega}\times\left(\boldsymbol{\omega}\times\boldsymbol{r}\right)$
per unit mass; if we take $\boldsymbol{\omega}=\omega\boldsymbol{e}_x$
the body force is $-\omega^2y\boldsymbol{e}_y$ and this is equivalent
to a potential energy of $-\omega^2y^2/2$ per unit mass.  We now
switch to coordinates $y^*$ with $y = Y_0 + y_0y^*$, corresponding to
an off-axis troposkein.  We find that the potential energy is
proportional to $y^* +
\frac{y_0}{Y_0}\left(y^*\right)^2/2$.  If
$Y_0$ and $y_0$ are such that $y_0\ll Y_0$ and $y^*={\mathcal O}(1)$,
the chain is a slightly perturbed catenary.  Dropping the star for
convenience and rescaling gives us

\begin{equation}
  F = \sqrt{1+y'^2}(y+\Lambda)\qquad
  G = \sqrt{1+y'^2}(y^2/2 + \lambda)
\end{equation}

Equation \ref{pertcan}, with $\phi=y^2/2$, becomes

\begin{equation}
  z'' - 4z'\tanh x - z = y^2/2 + \lambda -y\cosh x.
\end{equation}
%
This equation is analytically intractable so we use numerical methods.
Here I consider a standard $y=\cosh x$ catenary fixed at endpoints
$(\pm 2, \cosh 2)$.  To find meaningful solutions to this equation,
the left endpoint is constrained to zero perturbation: $z(-2)=0$.
Next, we find values of $\lambda$ and initial slope $z'(-2)$ that
jointly ensure $z(2)=0$ and that the chain has the same length as the
unperturbed solution.  Straightforward numerical optimization
techniques show that $z'(0)\simeq 0.662$ and $\lambda = -2.762$ satisfy
the constraints.  Figures~\ref{trop} and~\ref{trop_perturb} show
different views of this numerical solution numerical simulation.  We
see that the catenary is distorted into a ``vee'' shape.

\begin{figure}[h]
\centering % File showoptimsspace-2.pdf created by euler_lagrange_perturbed.Rmd
\includegraphics[width=0.9\textwidth]{euler_lagrange_perturbed_files/figure-latex/showoptimspace-2.pdf}
\caption{A uniform flexible chain in a uniform gravitational field
  (black; catenary) and in a slightly perturbed gravitational field (red)}
\label{trop_perturb}
\end{figure}

\begin{figure}[h]
\centering % File showoptimsspace-1.pdf created by euler_lagrange_perturbed.Rmd
\includegraphics[width=0.9\textwidth]{euler_lagrange_perturbed_files/figure-latex/showoptimspace-1.pdf}
\caption{The perturbation: that is, the difference between the red and
  black curves of figure~\ref{trop_perturb}}
\label{trop}
\end{figure}

\subsection{Black holes}

In the theory of general relativity~\cite{misner1973}, the vacuum
field equations for a stationary point mass are solved by the
Schwarzschild metric that gives the squared interval $ds^2$ in terms
of spherical coordinates $(r,\theta,\phi)$:

\begin{eqnarray}\label{schwarzschild}
  ds^2=
  -dt^2(1-2M/r) + \frac{dr^2}{1-2M/r} + r^2d\theta^2 + r^2\sin\theta d\phi^2
  \end{eqnarray}

In equation~\ref{schwarzschild}, $M$ is the mass of the black hole (it
is conventional to use geometrized units so the mass has the
dimensions of length; $2M$ is known as the Schwarzschild radius).
Black holes are known to exhibit counterintuitive behaviour near the
event horizon~\cite{allen1990} and one consequence of the curvature of
spacetime in the vicinity is that light inextensible string under
tension can adopt curved configurations~\cite{hankin2021}.  Such taut
inextensible strings minimize their proper length, given by
equation~\ref{proper}.

\begin{equation}\label{proper}
  \int_{p_1}^{p_2}\sqrt{r^2 + \frac{r'^2}{1-2M/r}}\,d\phi
\end{equation}

To find the configuration $r=r(\phi)$ of a taut string we minimize
functional $F=\sqrt{r^2 + \frac{r'^2}{1-2M/r}}$.  The Euler-Lagrange
equations yield

\begin{equation}\label{eulerstring}
  r''(\phi) = (r-2M) + \frac{(2r-3M)r'^2}{r(r-2M)}
\end{equation}


Following Einstein~\cite{einstein1915}, we consider
equation~\ref{schwarzschild} as a slight perturbation of classical
Newtonian physics.  First, we observe that in the limit
$M\longrightarrow 0$ we recover $F=\sqrt{r^2+r'^2}$, whence $r'' = r +
2r'^2/r$.  This has solution $r=r_0\sec(\phi-\phi_0)$, a familiar
straight line but expressed in polar coordinates.  To first order in
$2M$:

\begin{equation}
\sqrt{r^2 + \frac{r'^2}{1-2M/r}}=\sqrt{r+r'^2} + 2M\frac{r'^2}{2r\sqrt{r^2 + r'^2}}
\end{equation}
  
so $F=\sqrt{r^2+r'^2}$ and $G=\frac{r'^2}{2r\sqrt{r^2+r'^2}}$.  Then

\begin{eqnarray}\label{newtonian_string}
\frac{d}{dx}\left(\di{F}{r'}\right) -\di{F}{r} &=& \frac{r^2r''- 2rr'^2 -r^3}{(r^2 +r'^2)^{3/2}}\nonumber\\
\frac{d}{dx}\left(\di{G}{r'}\right) -\di{G}{r} &=&
\frac{d}{dx}\left(\frac{\partial}{\partial r'}\left(\frac{r'^2}{2r\sqrt{r^2+r'^2}}\right)\right)
- \frac{\partial}{\partial
  r}\left(\frac{r'^2}{r\sqrt{r^2+r'^2}}\right)\nonumber\\
&=& \frac{(2r^2-r'^2)(rr''-r'^2)}{2(r^2+r'^2)^{5/2}}
\end{eqnarray}

Equation~\ref{newtonian_string} gives $r'' = r + 2r'^2/r$ in the
unperturbed case from which we recover the familiar Euclidean
straight-line solution from Newtonian physics:
$r=r_0\sec(\theta-\theta_0)$.  The left hand side of
equation~\ref{factorized_eqn} is then

\begin{eqnarray}
&{}& \left(z\di{}{r} + z'\di{}{r'} + z''\di{}{r''}\right)
\left(\frac{d}{dx}\di{F}{r'}
-\di{F}{r}\right)\nonumber\\
&=&  \left(z\di{}{r} + z'\di{}{r'} + z''\di{}{r''}\right)
\left(\frac{r^2r''- 2rr'^2 -r^3}{(r^2 +r'^2)^{3/2}}\right)\nonumber\\
&=& 
 z  \left(\frac{(r^2 - 2r'^2)(rr''-r'^2)}{(r^2 + r'^2)^{5/2}}\right)\nonumber\\
 &{}& + z' \left(\frac{rr'(r^2 - 2r'^2 + 3rr'')}{(r^2 + r'^2)^{5/2}}\right)
      + z''\left(\frac{r^2    (r^2+r'^2)       }{(r^2 + r'^2)^{5/2}}\right)
\end{eqnarray}

Equation~\ref{factorized_eqn} in full becomes

\begin{eqnarray}
 &{}&z   \left((r^2 - 2r'^2)(rr''-r'^2)\right)\nonumber\\
+&{}&z'  \left(-rr'(r^2 - 2r'^2 + 3rr'')\right)\nonumber\\
+&{}&z'' \left(r^2    (r^2+r'^2)       \right)\nonumber\\
&{}&\qquad =\qquad -(2r^2-r'^2)(rr''-r'^2)/2
\end{eqnarray}
\begin{eqnarray}
  &{}&
  z   \left((r^2 - 2r'^2)(rr''-r'^2)\right)
  +z'  \left(-rr'(r^2 - 2r'^2 + 3rr'')\right)
  +z'' \left(r^2    (r^2+r'^2)       \right)\nonumber\\
 &=& -(2r^2-r'^2)(rr''-r'^2)/2
\end{eqnarray}

Now we are free to substitute.  We have $r=r_0\sec(\theta-\theta_0)$
but are given $r_0=1$ and will take $\theta_0=0$ for simplicity.  Thus
$r'= r_0\sec\theta\tan\theta$ and
$r''=\sec\theta+\sec\theta\tan^2\theta$.  Substituting and simplifying
gives:

\begin{equation}\label{string_ODE}
z'' - 4z'\tan \theta + z(1-2\tan^2\theta) = {\scriptstyle\frac{1}{2}}\tan^2\theta - 1
\end{equation}

This does not appear to possess a simple analytical solution but
numerical methods may be used.  We have two degrees of freedom that
correspond to the specified radius at $p_1$ and $p_2$.

To fix ideas we consider the Earth's orbit around the sun, taking it
to be a nominal circle of radius $r_0 = \qty{1.5e11}{\m}$,
and use this as the unit of length.  If we take the Schwarzschild
radius of the Sun to be $\qty{3e3}{\m}$, we find that the
nondimensionalised value of $2M$ is $10^{-8}$, and the linear
approximation is reasonable.

\begin{figure}[h]
\centering ng % File showblack-1.pdf created by euler_lagrange_perturbed.Rmd
\includegraphics[width=0.9\textwidth]{euler_lagrange_perturbed_files/figure-latex/showblack-1.pdf}
\caption{Linear perturbation from a Euclidean straight line caused by slightly curved spacetime 
  near a central point mass}
\label{showblack}
\end{figure}
 
Taking $p_1=(1,-\pi/4)$, $p_2=(1,\pi/4)$ then figure~\ref{showblack}
shows the deviation from the classical Newtonian solution due to
general relativity.  We see that the maximum deviation is about
$2M\cdot 0.7886$, or about $\qty{2.4}{\km}$.  Further, we may exploit
the linearity of equation~\ref{factorized_eqn}.  If we replace the Sun
with the Earth (of Schwarzschild radius $\qty{8.87}{\mm}$) in the
physical setup, then the perturbation is again $2M\cdot 0.7886$, or
about $\qty{7}{\mm}$; a result difficult to obtain any other way.


\section{Conclusions}

Using the Euler-Lagrange equation to minimize a slightly perturbed
functional appears to be a useful technique that can yield novel
insight in the context of classical variational problems.  I apply
this perturbation method to three physical problems and find linear
theory.  In each case, the linearised perturbation furnishes insight
into the problems considered.  In the cases of the hanging chain and
light inextensible string, the insights obtained by the linear
perturbation techniques presented here appear to be difficult to
obtain any other way.

\bibliography{sn-bibliography}% common bib file
%% if required, the content of .bbl file can be included here once bbl is generated
%%\input sn-article.bbl

\end{document}
